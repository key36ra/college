\documentclass[11pt,a4j]{jsarticle}
\title{光回折}
\author{1413176 三村幸祐}
\date{2016/10/12 \, 2016/10/19}
\usepackage{booktabs}

\pagestyle{empty}

\begin{document}
  
%  \tableofcontents
%  \newpage
  
 \section{目的}
  普段私たちがレーザーと呼ぶものは光の一種であるが、レーザー光は自然光と異なり特別な性質をいくつか持ち合わせている。代表的なものに、
  非常によい指向性・単色性・収束性・コヒーレントなどの性質があげられる。
  
  今回の実験の目的は、その指向性・コヒーレント性に着目し、大まかに以下の内容でレーザーの性質を見ていくことである。
  \begin{enumerate}
   \item レーザー光の断面の光強度分布を測定することで、そのビーム径や拡がり角を求める。
   \item スリットとピンホールを用いて、光のフラウンホーファー回折を定量的に調べる。
  \end{enumerate}

  
  
 \section{原理}
  \subsection{ガウスビームとビーム径}
  レーザー光の伝搬はガウス関数というものにしたがっている。
  
  レーザー光は概念的に光共振器という曲率半径Rの鏡を距離Lはなして向かい合わせた装置で光を増幅して放射したものである。
  そのとき最も光断面が小さくなるときの半径(ビーム径)$\omega_0$は次の式で与えられる。
  \begin{equation}
   \omega_0^2 = \frac{L}{k} \sqrt{\frac{2R}{L} - 1}
  \end{equation}
  
  さらに最小ビーム径から距離zにおけるビーム径$\omega(z)$は
  \begin{equation}
   \omega(z) = \omega_0 \sqrt{1+ \frac{4z^2}{k^2 \omega_0^4}}
  \end{equation}
  で表される。
  
  またレーザの断面強度分布$I(r)$は(rは断面円の中心からの距離)、
  \begin{equation}
   I(r) =  I_0 exp(-\frac{2r^2}{\omega(z)^2})
  \end{equation}
  となる。
  
  拡がり角$\theta$は2点でwを求めたとき、それらを$\omega_1,\omega_2$とすると、
  \begin{equation}
   tan(\theta) = \frac{\omega_2 - \omega_1}{z_2 - z_1}
  \end{equation}
  から求める。
  
  実際の実験では$\omega_0$を直接測定することは出来ないため、上の$\theta$の測定結果から
  \begin{equation}
   \omega_0 = \frac{\lambda}{\pi \theta}
  \end{equation}
  として、これを測定結果とする。


  
  \subsection{フラウンホーファー回折}
  幅2aのスリットを通してレーザーを放射すると、スクリーンには次の関数による縞模様が現れる。
  \begin{equation}
   I(x) = I_0 \frac{(sin(\frac{2\pi ax}{\lambda z}))^2}{(\frac{2\pi ax}{\lambda z})^2}
  \end{equation}
  
  一方、半径aのピンホールを通した場合は、
  \begin{equation}
   I(S) = {\frac{2J_1 (kaS)}{kaS}}^2
  \end{equation}
  となる。ただし、S=r/zで$J_1 (kaS)$は第1ベッセル関数であり、最初にその関数が極小になる点は
  \begin{eqnarray}
   kaS = 1.22 \pi \\
   a = \frac{0.61 \lambda z}{r_0}
  \end{eqnarray}
  となる。



  
  
  
  
 \section{測定手順}
 波長$\lambda$=663nmのヘリウムネオンレーザーを光源としてレーザーをCCDカメラに照射し、その画像データから光断面強度を測定。
 ここで使用したCCDカメラはソニー製 XC-77で、撮像面積8.8×6.6mm、有効画素数768×493である。
  \subsection{レーザー光のビーム径と拡がり角}
  光源から放射されたレーザーを直接CCDカメラに入射させ、断面の光強度分布を測定し、ビーム径と拡がり角を測定。
  \begin{enumerate}
   \item CCDカメラと光源の間にフィルターをはさみ、CCDで読み込める程度の大きさに強度を弱めた。
         ただし、CCDと光源間の距離z=1mとして、さらにフィルターによりレーザー光の強度は、
         0.000188\verb|%|(0.3\verb|%|×1, 10\verb|%|×2, 25\verb|%|×2)に弱めている。
   \item 上で測定したレーザー光の光強度断面のガウス関数において、最も強い強度を$I_0$、またその時のxを0点として分布グラフを作成。また$I_0$の$e^{-2}$倍となるx座標を
         求めて、その点を$\omega_0$とした。
   \item 次はレーザーの拡がり角を求めるために、上の1,2の操作をz=4mに対してもう一度行った。
  \end{enumerate}

  
  \subsection{スリットとピンホールの回折パターン}
  回折用のスリットとピンホールについてフラウンホーファー回折パターンを測定し、スリット・ピンホールの形状・大きさなどと関連付ける。
  \begin{enumerate}
   \item 幅2a=0.1mmのスリットを用いて回折パターンを測定。ここでスリットとスクリーンとの距離z=70.5cmとした。
   \item 未知の幅をもつスリットの回折パターンから、その幅を特定。スリットとスクリーンの距離はz=50.5cmとした。
   \item 未知半径のピンボールについて、2と同様の操作を行った。スリット・スクリーン間の距離は2と同じく50.5cmにした。
  \end{enumerate}

  
  
  
 \section{結果と考察}
  \subsection{レーザー光のビーム径と拡がり角}
  図1にレーザービームの断面光強度の測定値と、その$\omega_0$から求めた理論値を、図2にz=4mでの光断面測定値を示した。

  図1の理論グラフの算出過程を以下に示す。
  \begin{eqnarray}
実験グラフより周囲の光&に&よる光強度I(x)_{offset} = 6を除き、I_0=75として、 \nonumber \\
I(x) &=& 75 exp(- \frac{ x^2}{0.93}) +6 \ \ \  (ただし、xの単位はmm) \nonumber 
  \end{eqnarray}
  
  図1よりz=1mの時の中心ピークの両側の最初の強度ゼロの2点間の距離から、ビーム径は$\omega$=0.963mmであった。
  また同図から測定値と理論グラフは一致しているといってよい。測定値がx=大、でもI(x)が0にならないのは実験室内の光をCCDが検知してしまったためである。また、このグラフは実験値が6よりも小さくならないことからオフセットとして6を定め、その値を無視したときの値から理論値も算出している。すると図1のように実験値と理論値は整合した。つまりこれが意味するところは、領域全体に同じような余分な光強度が発生しているということである。これが雑音というもので、雑音を除いて計算した理論値が実験値と近くなることから、光強度の測定誤差には雑音が最も大きな影響を与えていると言うことがわかる。
  
  さらに広がり角と最小ビーム半径について計算する。図2よりz=4mの時のビーム径は$\omega$=3.67mm。したがって(4),(5)式より、
  \begin{eqnarray}
   広がり角  \theta = 9.03 × 10^{-4} rad \nonumber \\
   最小ビーム径  \omega_0 = 0.223mm \nonumber
  \end{eqnarray}
  となる。この値は、100m先でも9.05cmにしか広がらない、という程度の指向性をもつことを意味する。これは実験値から近似した広がり角、最小ビーム径の実験値と言えるが、これは式(2)のzに100m、$\omega_0$に今回求めた値を入れた計算値$\omega(100)$=9.04cm
  とも0.1\verb|%|の誤差のみであるため、実験値と計算値の整合から今回の指向性(拡がり角)の測定は精度が高い。
  
  またある文献によると([1])、波長0.6$\mu$mの黄赤色のレーザーは拡がり角が$3×10^{-4}$rad程度で、100m進んでも3cmにしか広がらない、とあるため今回の実験の測定はこれと比較すると、
  低い指向性となっている。これは、室内の余計な光を検出してしまって、十分に減衰した点(光強度が$e^{-2}$倍になる点)の測定幅が広くなってしまったためだと考える。精度を高くするには上記のように雑音を取り除くことが必要になり、たとえばある特定のそのレーザー光の波長域のみを調べるために、CCDカメラ付近に色ガラスフィルターなどを複数枚用意するなどして、レーザーの指向性などの特性を調べるためには、対処する必要がある。
  
  \clearpage
  
  \subsection{スリットとピンホールの回折パターン}
  まずスリット幅が0.1mmの時の回折パターンの測定値と理論値を図3に、未知幅のスリット測定を図4に、ピンホールの測定を図5に示した。また、その図3のスリット回折パターンの理論グラフの算出過程を以下に示す。

\begin{eqnarray}
今回ここではx=-1.5~2mmの範囲&の&光強度の平均値を計算し中心の光強度I_0 = 6368 とした。 \nonumber \\
I(x) &=& 6368 * \frac{2.2}{x^2} (sin(0.67x))^2 \nonumber \\
       &=& \frac{14100}{x^2} (sin(0.67x))^2 \ \ \ (ただし、xの単位はmm) \nonumber
\end{eqnarray}

  
  図3についてのピーク光強度は、中心ピークから$\pm 1.5mm$範囲の光強度の平均から求めた。そこに式(6)から計算した理論値を載せたのが図3だが、
  理論と測定値はそのピークの工夫は無視するとして、一致している。今回は光の干渉性を見る実験であった。そのためビーム径と広がり角を求めた実験に対して、オフセット(雑音)を含めた互いの光が強めあって(弱めあって)いるため、CCDに移る光強度分布はx=∞で限りなく0に近づいている。その形は極致を幾度も持ちながら減衰していく様子が確かに理論のsincx関数と一致することも確認できる。

このように中心から離れる際の挙動が理論と一致することが確認できた。しかし、中心付近の実験値を見てみるとばらつきが大きい。ここからわかるのはいくつも強まる点が中心付近に発生しているということである。理論的には1つの頂点ができるが、それはホイヘンスの定理によってスリット等価付近の微小区間で連続的に球面波の波源が幾何学的に整列しているために、数学的にしめされるものである。つまりこの実験値のように中心のばらつきがあらわれるのは、使用するスリットの穴が厳密な点対称、もしくは線対称からかけはなれているということが言える。このように多少雑のような物理現象になってしまうが、この干渉性関しては、それでも理論値と実験値は近似を得られることが分かる。これはレーザーの高い干渉性によるもので、今回の実験でそれが確認できる。
  
  また同様にして今度は図4,5と式(6)、もしくはピンホールには(7)を用いてそのスリット幅やピンホール半径を求めると、
  \begin{eqnarray}
   スリット幅 2a = 44.9 \mu m \nonumber \\
   ピンホール幅 a = 85.5 \mu m \nonumber
  \end{eqnarray}
  となる。この値も前のスリット(幅2a=0.1mm)とオーダーが近いものを使用しているため、確かに理論から得られた値がそのスリットの(ピンホールなども)情報を抜き取ることができるということを確認した。
  
  \clearpage
  
  
 \section{計算練習課題}
 \begin{enumerate}
  \item 式(1),(2)より
  \begin{eqnarray}
   \omega_0^2 k &=& 0.2 \sqrt{\frac{2 * 1}{0.2} - 1} \nonumber \\
   \omega_0     &=& 0.77 k^{-\frac{1}{2}} \nonumber \\
   \omega(0.1)  &=& \omega_0 \sqrt{1 + \frac{4 * 0.01}{0.6^2}} \nonumber \\
                &=& 0.82 k^{-\frac{1}{2}} \nonumber
  \end{eqnarray}
  
  \item 
  \begin{eqnarray}
   F(\mu) &=& \int_{-a}^a e^{-j\mu x} dx \nonumber \\
          &=& \frac{j}{\mu}(e^{-ja\mu} - e^{ja\mu}) \nonumber
  \end{eqnarray}


 \end{enumerate}


  
  
 \section{参考文献}
  [1] 霜田光一,岩波書店,"レーザー物理入門",p2
  
  
\end{document}