\documentclass[11pt,a4j]{jsarticle}
\title{}
\author{1413176 三村幸祐}
\date{2016/12/14}
\usepackage{booktabs}
\usepackage[dvipdfmx,hiresbb]{graphicx}
%\pagestyle{empty}

\begin{document}
  
% \tableofcontents
  
% \newpage
  
  
% 実験の背景と意義、レポートの構成などを記述
 \section{はじめに}
  デジタル信号は0と1の信号の組み合わせでできている。各信号が0と1の2値であるが、現代のデジタル機器では幅広い数値を表現することができる。つまり離散的な値の組み合わせで、見掛け上連続的な幅を表現できるということになる。
  
  ここでは、このコンピュータのシステムを演算装置(ALU)から、Z80コンピュータシステムやI/O制御までを通して理解することを目的とする。
  
  レポートはこの実験全体を2分し、以下のように各レポートで実験結果をまとめる。
  \begin{itemize}
  \item 1部: 演算装置(ALU:Arithmetic Logic Unit)
  \item 2部: Z80コンピュータシステム、I/O制御
  \end{itemize}
  
  そして、このレポートはそのうちの第1部である。
  
% 参考書などを用いて、目的と原理を記述
 \section{実験の目的と原理}
  
  
  
% 実験の手順を順次説明
 \section{実験内容}
  
  
  
% 図や表による実験データの添付、またそれらの説明を記載
 \section{実験結果}
  
  表\ref{tab:1junbi}にはALUの演算入出力応答について、事前に予想した準備課題1を示す。
  
  \begin{table}[htb]
  \begin{center}
    \caption{準備課題1}
    \begin{tabular}{|c|c|c|c|c|c|c|c|c|} \hline
課題 & \multicolumn{2}{|c|}{入力A} & \multicolumn{2}{|c|}{入力B} & \multicolumn{2}{|c|}{演算出力} & 桁上がり & 桁借り\\
 & \multicolumn{2}{|c|}{(A7~A0)} & \multicolumn{2}{|c|}{(B7~B0)} & \multicolumn{2}{|c|}{(07~O0)} & (Cout) & (Bout) \\ \cline{2-7}
番号 & 2進表示 & 10進表示 & 2進表示 & 10進表示 & 2進表示 & 10進表示 & キャリー & ボロー \\ \hline
 & 00001010 & 10 & 00001010 &10  &00010100  & 20 & 0 &  \\ \cline{2-8}
1(加) & 01010101 & 85 & 01101010 &106  &10111111  & 191 & 0 & × \\ \cline{2-8}
 & 10000000 & 128 & 01000000 &64  & 11000000 & 192 & 0 &  \\ \hline
 & 00011000 &24  & 00001000 &8  &00010000  & 16 &  & 0 \\ \cline{2-7}\cline{9-9}
1(減) & 01101001 & 105 &00110110  &54  & 00110011 & 51 & × & 0 \\ \cline{2-7}\cline{9-9}
 & 10000000 & 128 &00010000  & 16 &01110000  & 112 &  & 0 \\ \hline
 & 10000001 & 129 &11000000  & 192 &01000001  &65  & 1 &  \\ \cline{2-8}
2(加) & 11000000 & 192 & 01100000 &96  &00100000  &32  & 1 & × \\ \cline{2-8}
 & 10000001 &  129& 10000001 & 129 &00000010  & 2 & 1 &  \\ \hline
 & 00001010 &10  &10100000  &160  &01101010  &106  &  & 1 \\ \cline{2-7}\cline{9-9}
2(減) & 00011110 &  30&01100100  & 100 & 10111010 &186  & × & 1 \\ \cline{2-7}\cline{9-9}
 & 10100000 & 160 &11000000  & 192 & 11100000 &224  &  & 1 \\ \hline
 &00001010  & 10 & 00010101 &21  & 00011111 &31  & 0 &  \\ \cline{2-8}
3(加) & 11110000 &-16  & 00110000 & 48 & 00100000 & 32 & 1 & × \\ \cline{2-8}
 & 10110000 & -80 &00011110  & 30 & 11001110 &-50  & 0 &  \\ \hline
 & 01001001 & 73 & 00011100 &28  & 00101101 &45  &  & 0 \\ \cline{2-7}\cline{9-9}
3(減) & 11110000 & -16 &11100000  & -32 &00010000  &16  & × & 0 \\ \cline{2-7}\cline{9-9}
 & 10100000 & 160 & 11000000 & 192 &11100000  &-32  &  & 1 \\ \hline
 &01001001  & 73 &01100100  & 100 & 10101101 & -83 & 0 &  \\ \cline{2-8}
4(加) &  10011000& -104 & 11010001 & -47 & 01101001 &105  & 1 & × \\ \cline{2-8}
 &  10001000&-120  &  10000100& -124 &00001100  & 12 & 1 &  \\ \hline
 &10001000  & -120 & 01000100 & 68 &01000100  & 68 &  & 0 \\ \cline{2-7}\cline{9-9}
4(減) &01010000  & 80 & 10110000 & -80 & 11100000 & -32 & × & 1 \\ \cline{2-7}\cline{9-9}
 &01001001  &73  & 10011100 & -100 &10101101  &-83  &  & 1 \\ \hline
 & 01001010 &74  &00000001  &1  &00100101  & 37 &  &  \\ \cline{2-7}
5(論) &01100000  & 96 & 00000011 & 3 &00001100  &12  & × & × \\ \cline{2-7}
 & 10001010 &-118  & 00000111 &5  & 00000100 &4  &  &  \\ \hline
 &01101001  &105  &00000001  & 1 &00110100  &52  &  &  \\ \cline{2-7}
5(算) & 11100011 & -29 & 00000011 &3  &11111100  &-4  & × & × \\ \cline{2-7}
 & 10001010 & -118 &00000111  & 5 & 11111100 &-4  &  &  \\ \hline
    \end{tabular}
    \label{tab:1junbi}
  \end{center}
 \end{table}
  
  
  表\ref{tab:1kadai}に実際のALUの応答結果を示す。
  
  
  \begin{table}[htb]
  \begin{center}
    \caption{実験課題1(動作検証表)}
    \begin{tabular}{|c|c|c|c|c|c|c|c|c|} \hline
課題 & \multicolumn{2}{|c|}{入力A} & \multicolumn{2}{|c|}{入力B} & \multicolumn{2}{|c|}{演算出力} & 桁上がり & 桁借り\\
 & \multicolumn{2}{|c|}{(A7~A0)} & \multicolumn{2}{|c|}{(B7~B0)} & \multicolumn{2}{|c|}{(07~O0)} & (Cout) & (Bout) \\ \cline{2-7}
番号 & 2進表示 & 10進表示 & 2進表示 & 10進表示 & 2進表示 & 10進表示 & キャリー & ボロー \\ \hline
 & 00001010 & 10 & 00001010 &10  &00010100  & 20 & 0 &  \\ \cline{2-8}
1(加) & 01010101 & 85 & 01101010 &106  &10111111  & 191 & 0 & × \\ \cline{2-8}
 & 10000000 & 128 & 01000000 &64  & 11000000 & 192 & 0 &  \\ \hline
 & 00011000 &24  & 00001000 &8  &00010000  & 16 &  & 0 \\ \cline{2-7}\cline{9-9}
1(減) & 01101001 & 105 &00110110  &54  & 00110011 & 51 & × & 0 \\ \cline{2-7}\cline{9-9}
 & 10000000 & 128 &00010000  & 16 &01110000  & 112 &  & 0 \\ \hline
 & 10000001 & 129 &11000000  & 192 &01000001  &65  & 1 &  \\ \cline{2-8}
2(加) & 11000000 & 192 & 01100000 &96  &00100000  &32  & 1 & × \\ \cline{2-8}
 & 10000001 &  129& 10000001 & 129 &00000010  & 2 & 1 &  \\ \hline
 & 00001010 &10  &10100000  &160  &01101010  &106  &  & 1 \\ \cline{2-7}\cline{9-9}
2(減) & 00011110 &  30& 11000000 & 192 & 01011110 & 94 & × & 1 \\ \cline{2-7}\cline{9-9}
 & 10100000 & 160 &11000000  & 192 & 11100000 &224  &  & 1 \\ \hline
 &00001010  & 10 & 00010101 &21  & 00011111 &31  & 0 &  \\ \cline{2-8}
3(加) & 11110000 &-16  & 00110000 & 48 & 00100000 & 32 & 1 & × \\ \cline{2-8}
 & 10110000 & -80 &00011110  & 30 & 11001110 &-50  & 0 &  \\ \hline
 & 01001001 & 73 & 00011100 &28  & 00101101 &45  &  & 0 \\ \cline{2-7}\cline{9-9}
3(減) & 11110000 & -16 &11100000  & -32 &00010000  &16  & × & 0 \\ \cline{2-7}\cline{9-9}
 & 10100000 & -96 & 11000000 & -64 &11100000  &-32  &  & 1 \\ \hline
 &01001001  & 73 &01100100  & 100 & 10101101 & -83 & 0 &  \\ \cline{2-8}
4(加) &  10011000& -104 & 11010001 & -47 & 01101001 &105  & 1 & × \\ \cline{2-8}
 &  10001000&-120  &  10000100& -124 &00001100  & 12 & 1 &  \\ \hline
 &10001000  & -120 & 01000100 & 68 &01000100  & 68 &  & 0 \\ \cline{2-7}\cline{9-9}
4(減) &01010000  & 80 & 10110000 & -80 & 10100000 & -96 & × & 1 \\ \cline{2-7}\cline{9-9}
 &01001001  &73  & 10011100 & -100 &10101101  &-83  &  & 1 \\ \hline
 & 01001010 &74  &00000001  &1  &00100101  & 37 &  &  \\ \cline{2-7}
5(論) &01100000  & 96 & 00000011 & 3 &00001100  &12  & × & × \\ \cline{2-7}
 & 10001010 &-118  & 00000101 &5  & 00000100 &4  &  &  \\ \hline
 &01101001  &105  &00000001  & 1 &00110100  &52  &  &  \\ \cline{2-7}
5(算) & 11100011 & -29 & 00000011 &3  &11111100  &-4  & × & × \\ \cline{2-7}
 & 10001010 & -118 &00000101  & 5 & 11111100 &-4  &  &  \\ \hline
    \end{tabular}
    \label{tab:1kadai}
  \end{center}
 \end{table}
 
 
 
  
% 実験結果に基づいて課題の回答及びその考察を詳しく記載
 \section{考察}
  
   \subsection{実験結果の検証と解析}
   
   
   \subsection{各条件におけるオーバフローの法則}
   
   
   \subsection{オーバフローの検出}
   
   
   
  
  
% まとめ
 \section{おわりに}
  
  
  
% 参考文献の列挙
 \section{参考文献}
  
  
  
\end{document}