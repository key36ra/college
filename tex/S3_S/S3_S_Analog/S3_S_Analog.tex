\documentclass[11pt,a4j,titlepage]{jsarticle}
\title{アナログ回路I}
\author{1413176 三村幸祐}
\date{2016/07/07 \, 2016/07/14}
\usepackage{booktabs}

\pagestyle{empty}

\begin{document}
 
 
  
 \section{目的}
  本実験ではバイポーラトランジスタに直流、または交流の電流を流したときにパラメータや特性を確認することを主な目的とする。
  2週に分けて実験を行ったが、まず第1週目に使用するトランジスタのパラメータ特性(hパラメータなど)を測定し、2週目にそのパラメータを用いて
  ある仕様の増幅回路設計を行い、その振る舞いを測定する。
  
 
 
  % 1週目の実験
 \section{特性測定(第1週目の実験)}

  \subsection{直流特性}
 
   \subsubsection{原理}
   直流特性の測定回路を図1に示す。
   この回路はエミッタ接地回路であるが、コレクタC、エミッタE、ベースBを用いて、測定する電圧値$V_1,V_2$
   はそれぞれ$V_{BE},V_{CC}$を表す。
   
   この実験ではトランジスタに直流を流したときにベース電流$I_B$に対して、コレクタ電流$I_C$として何倍の出力が得られるかを
   表すパラメータ$h_{FE}=I_C/I_B$を求めることが1つ、目的としてある。
   
   また2つ目の目的としては、トランジスタを局所的に見た場合のダイオード的性質を確認することにある。
   今回用いるようなバイポーラトランジスタはnpn接合となっており同順でエミッタE、ベースB、コレクタCというように並んでいる。
   したがってここで測定するベース-エミッタ間のみを見てみると、p→nと言う順方向電圧のかかったダイオード構造と同じである。
   したがってベース電流$I_B$と$V_{BE}$を関係づけると、ダイオードの静的特性が現れるはずである。これを確認するのがここでのもうひとつの目的である。
   
   したがって、ベース電流とバイアス電圧の間には
    \begin{eqnarray}
     I_B &=& I_0 exp(\frac{qV_{BE}}{NkT}) \\
     ln(I_B) &=& \frac{q}{NkT} V_{BE} + ln(I_0)
    \end{eqnarray}
   の関係が成り立つはず(図\ \ 参照)。
   
   \newpage
   
   まとめる。
   \begin{itemize}
    \item 入力に対する出力を表すパラメータ
	  \begin{equation}
           h_{FE} = I_C / I_B
          \end{equation}
          を求める。
    \item $V_{BE},I_B$にダイオードの静的特性を確認する。
   \end{itemize}

  
  \newpage
  
   \subsubsection{測定手順}
  使用する回路素子、測定装置を次に示す。
  \begin{itemize}
   \item トランジスタ 2SC1815
   \item $R_C$ 1k$\Omega$
   \item $R_B$ 100k$\Omega$
   \item デジタルマルチメータ LIKE, 8845A
   \item 直流電源 GWINSTEK, GPD-3303S
  \end{itemize}
  
  \begin{enumerate}
   \item $R_C,R_B$の抵抗値をDMMを用いて測定。
   \item 図\ \ の測定回路を組立。
   \item $V_1,V_2$の値を計測。
  \end{enumerate}
  
  
  
   \subsubsection{結果と考察}

   今回使用した$R_B,R_C$のデジタルマルチメータ測定値は以下のようである。
 
 \begin{table}[htb]
  \begin{center}
    \caption{$R_B,R_C$の測定値}
    \begin{tabular}{cc} \toprule
$R_B/k\Omega$ & $R_C/k\Omega$ \\ \midrule
98.22 &	0.9994\\ \bottomrule
    \end{tabular}
    \label{tab:price}
  \end{center}
\end{table}
 
 さらに表2には$I_B,I_C,V_{BE},V_{CC}$それぞれの関係を示す。
 \newpage
 \begin{table}[htb]
  \begin{center}
    \caption{直流特性}
    \begin{tabular}{cccc} \toprule
$I_B$/mA & $I_c$/$\mu$A & $V_{BE}$/V & $V_{CC}$/V \\ \midrule
0.066	&	0.006	&	0.4998	&	12.00	\\
0.178	&	0.026	&	0.5501	&	12.00	\\
0.889	&	0.161	&	0.5998	&	11.97	\\
6.241	&	1.187	&	0.6500	&	11.80	\\
0.448	&	0.077	&	0.5803	&	11.99	\\
1.298	&	0.240	&	0.6102	&	11.96	\\
10.00	&	1.908	&	0.6614	&	11.72	\\
20.00	&	3.821	&	0.6775	&	12.00	\\
30.01	&	5.695	&	0.6881	&	12.00	\\
40.01	&	7.551	&	0.6974	&	12.00	\\
50.00	&	9.281	&	0.7057	&	12.00	\\
60.01	&	10.90	&	0.7145	&	12.00	\\
15.01	&	2.864	&	0.6696	&	12.00	\\ \bottomrule
    \end{tabular}
    \label{tab:price}
  \end{center}
\end{table}
 
 $V_{BE}-I_B$特性を図\ \ に、$I_B-I_C$特性を図\ \ に示す。
 
 $I_B-I_C$からhパラメータを求めたい。図\ \ より、
 \begin{equation}
  h_{FE}(40\mu A) = 186 \nonumber
 \end{equation}
 となる。この値は次の交流特性との比較することになる(交流特性の項を参照)。
 
 また図\ \ を見るとここでの2つめの目的であったダイオードの静的特性が見られるのがわかる。
 これは原理で述べた通り、pn接合が局所的なダイオードを生成しているために見られる現象である。
 
 式(2)からでは理論的にfittingすることは難しいため、ここでは式(3)の直線化したグラフを図\ \ に示した。
 この図より、
 \begin{eqnarray}
  N &=& \frac{q}{kT}\frac{1}{34.6} \nonumber \\
    &\doteq& 1.12 \nonumber
 \end{eqnarray}
 として直線近似される。この直線は定性的に理論式(3)を満たすとともに、計算した理想係数Nの値の理論値は
 1~2となることもわかっている。理想係数はpn接合境界面における結晶性などによって変化する値であり、
 結晶性がよいほど1になる[1]。つまり、今回のトランジスタは結晶性の良い物質だといえる。


  
  \clearpage
  
  \subsection{交流特性}
 
   \subsubsection{原理}
   交流特性の測定回路図は図\ \ に示す。
   
   この節では交流を流した際の、$I_b$に対する$h_{fe},h_{ie}$の挙動を調べる。
   式(1)より、$h_ie$は
   \begin{eqnarray}
    \frac{dI_B}{dV_{BE}} &=& \frac{q}{NkT} I_0 exp(\frac{qV{BE}}{NkT}) \nonumber \\
    &=& \frac{q}{NkT} I_B \\
    h_{ie} &=& \frac{dV_{BE}}{I_B} \nonumber \\
         &=& \frac{NkT}{q} \frac{1}{I_B}
   \end{eqnarray}
   とかける。
   ただし、図1,(b)にあるように$I_C$は$I_B$に上昇するに連れて、いつまでも線形で大きくなるとは限らない。これはpn接合の性質によるもので、飽和領域という領域まで$I_B$を挙げることで、徐々に
   $h_{fe}$の値は小さくなっていく。同図中の3つの領域によって、$I_b-h_{fe}$のグラフも3つの領域を持つことは予想できる。
   
   また図1の(c)より、
   \begin{equation}
    h_{fe} = \frac{dI_C}{dI_B} = const 
   \end{equation}
   となるはずである。$h_{ie},h_{fe}$の概形は図\ \ ,\ \ に示す。
  
  
  \newpage
  
   \subsubsection{測定手順}
  使用する回路素子、測定装置は次に示す。ただし、$R_C,R_B$は直流特性と同様のものを使用。
  \begin{itemize}
   \item トランジスタ 2SC1815
   \item $R_C$ 1k$\Omega$
   \item $R_B$ 100k$\Omega$
   \item $C_{FG}$ 1$\mu$F
   \item $R_{PS}$ 220k$\Omega$
   \item ファンクションジェネレータ WAVEFACTORY, WF1973
   \item オシロスコープ Tektronix, TBS1154
   \item LCRメータ HIOKI, IM3523
  \end{itemize}
  
  \begin{enumerate}
   \item LCRメータで$C_{FG}$を測定。
   \item DMMで$R_{PG}$を測定。
   \item 図\ \ の実験回路の組立。
   \item DCMで$V_4$を測定。
   \item オシロスコープで$v_1(pp),v_2(pp),v_3(pp)$を測定。
  \end{enumerate}

  
  
   \subsubsection{結果と考察}
   
   
 
 \begin{table}[htb]
  \begin{center}
    \caption{$C_{FG},R_{PS}$の測定値}
    \begin{tabular}{cc} \toprule
$C_{FG}$/$\mu$F & $R_{PS}$/k$\Omega$ \\ \midrule
1.055 & 220.4 \\ \bottomrule
    \end{tabular}
    \label{tab:price}
  \end{center}
\end{table}
 
   
 
 \begin{table}[htb]
  \begin{center}
    \caption{交流回路各点の電圧と電流}
    \begin{tabular}{ccccc} \toprule
$V_4$/V	&	$I_B$/$\mu$A	&	$v_1$/mV	&	$v_2$/mV	&	$v_3$/mV	\\ \midrule
11.68	&	10.00	&	352	&	10.0	&	768	\\
11.52	&	15.00	&	512	&	9.8	&	1110	\\
11.35	&	20.02	&	720	&	10.6	&	1560	\\
11.20	&	24.98	&	820	&	10.0	&	1780	\\
11.95	&	29.99	&	960	&	10.0	&	1800	\\
11.94	&	35.01	&	1120	&	10.0	&	2080	\\
11.93	&	39.99	&	1260	&	10.2	&	2300	\\
11.92	&	44.98	&	1360	&	9.8	&	2440	\\
11.91	&	50.04	&	1480	&	10.0	&	2620	\\
11.90	&	55.05	&	1640	&	10.0	&	2640	\\
11.89	&	60.02	&	1860	&	10.0	&	2180	\\
11.12	&	27.51	&	920	&	10.0	&	1920	\\ \bottomrule
    \end{tabular}
    \label{tab:price}
  \end{center}
\end{table}

 
 \begin{table}[htb]
  \begin{center}
    \caption{交流特性}
    \begin{tabular}{ccccc} \toprule
$i_b(pp)$/$\mu$A	&	$i_c(pp)$/mA	&	$v_{be}(pp)$/mV	&	$h_{ie}$/$\Omega$	&	$h_{fe}$	\\ \midrule
3.4826	&	0.7685	&	10	&	2871.8	&	220.69	\\
5.1133	&	1.1107	&	9.8	&	1916.6	&	217.21	\\
7.2229	&	1.5609	&	10.6	&	1467.5	&	216.11	\\
8.2472	&	1.7811	&	10	&	1212.5	&	215.96	\\
9.6727	&	1.8011	&	10	&	1033.8	&	186.20	\\
11.302	&	2.0813	&	10	&	884.82	&	184.15	\\
12.725	&	2.3014	&	10.2	&	801.56	&	180.85	\\
13.747	&	2.4415	&	9.8	&	712.86	&	177.60	\\
14.967	&	2.6216	&	10	&	668.13	&	175.15	\\
16.596	&	2.6416	&	10	&	602.55	&	159.17	\\
18.836	&	2.1813	&	10	&	530.89	&	115.80	\\
9.2654	&	1.9212	&	10	&	1079.3	&	207.35	\\ \bottomrule
    \end{tabular}
    \label{tab:price}
  \end{center}
  \end{table}
  
  \clearpage
  
  $I_b-h_{fe},I_b-h_{ie}$のグラフをそれぞれ図\ \ ,\ \ に示した。
  
  原理では$h_{fe}$は$I_b$によらず一定となるはずであるが、2段の階段状になっている。これは同原理でも述べたように、$h_{fe}$は$I_b$により常に一定とは限らず3つの領域をとるはずである。
  この点では2つの階段で遮られて3つの$h_{fe}$の一定領域が生成されているため、これは図\ \ に示したような原理道理の図となっている。
  
  また$h_{ie}$のグラフではfittingしやすいように両辺を逆数で示した図\ \ を示した。これによると線形のグラフの傾きが0.029であるため
  \begin{eqnarray}
   \frac{q}{NkT} = 0.029 \nonumber \\
   \frac{NkT}{q} \cong 34.5 \nonumber
  \end{eqnarray}
  となり前節(直流特性)の図\ \ の実験データとも整合する。したがって$h_{ie}$のグラフも原理どおりの結果が得られた。
  
  
  
  さて、$h_{oe}$の増幅器に与える影響について少し考えてみる。
  付録1で求めた例を用いると、$h_{oe}=0.13$mS(ジーメンス)となってる。これは今回の$V_{CE}$の測定範囲~2Vで考えるとこの範囲における$I_C$の変化というのは0.2mA程度ということになる。
  これは図\ \ よりわかる$I_B$による変化10mAに比べるとわずか2\verb|%|程度となっている。そのため増幅器では$V_{CE}$による$I_C$の変化を示した、$h_{oe}$パラメータは他のhパラメータに比べて
  無視できるほど小さいと言える。
  
  
 
 
   
   \clearpage
  
  
  
  
  
  
  % 2週目の実験
 \section{増幅器回路の特性測定(第2週目の実験)}
  
  \subsection{回路製作・動作点の確認}
 
   \subsubsection{原理}
   
   細かな設計手順については付録の2(準備課題2)を参照されたい。
   
   増幅器の測定回路図を図\ \ に示す。
   
   $h_{ie}=Z_i=1200\Omega$とおおよそなるようなhパラメータを用いて設計した。
   図\ \ から上の条件を満たす時の$I_B$は24.98$\mu$Aであった(図\ \ にその時の$h_{ie},h_{fe}$を示した)。
   
   また図\ \ よりこの時の$I_C$を$I_B$の値から求めてみると、
   \begin{equation}
    V_{BE} = 0.685 V
   \end{equation}
   となった。
   この値は同様にダイオードの立ち上がり電圧値に近い。これは前述したように、トランジスターは局所的にそのnpn接合の性質から
   ダイオードの特性が現れることは説明できる。

  
  
  \newpage
  
   \subsubsection{測定手順}
  理想仕様は次の通り。
  \begin{description}
   \item[(1)] 電圧利得Av 150倍
   \item[(2)] $v_0$:5V(pp)
   \item[(3)] 1kHz以上の周波数において、電圧利得が一定
   \item[(4)] $Z_i$:1200$\Omega$
  \end{description}
  
  また実験室内の素子の中で最も上の仕様を満たすと思われるものを選んだ。
  \begin{itemize}
   \item $R_B$ 100k$\Omega$
   \item $R_1$ 27k$\Omega$
   \item $R_2$ 8.2k$\Omega$
   \item $R_E$ 330$\Omega$
   \item $R_C$ 820$\Omega$
   \item $C_E$ 330$\mu$F
   \item $C_C$ 100$\mu$F
  \end{itemize}
  
  \begin{enumerate}
   \item DMMで$R_1,R_2,R_E,R_C$の抵抗値を、LCRメータで$C_E,C_C$の容量を測定。
   \item 図\ \ の増幅器の特性測定回路を組立。
   \item DCMで$V_{CC},V_C,V_E,V_B$を測定。
   \item $V_{BE}$を求め、図\ \ よりそのときの$I_B$を計算。
   \item $V_{CE},I_C$を求め、動作点の$V_{CEQ},I_{CQ}$として記録。
  \end{enumerate}
  

  
  
   \subsubsection{結果と考察}
   
   使用素子の測定値と回路図各点の電圧の測定値を次に示す。
 
 \begin{table}[htb]
  \begin{center}
    \caption{増幅器用、使用素子の測定値}
    \begin{tabular}{ccccccc} \toprule
Rb/kΩ	&	R1/kΩ	&	R2/kΩ	&	Re/Ω	&	Rc/Ω	&	Ce/μF	&	Cc/μF	\\ \midrule
98.04	&	26.94	&	8.199	&	300.56	&	820.26	&	325.98	&	82.385	\\ \bottomrule
    \end{tabular}
    \label{tab:price}
  \end{center}
\end{table}

 
 \begin{table}[htb]
  \begin{center}
    \caption{増幅器回路各点、電圧の測定値}
    \begin{tabular}{cccc} \toprule
$V_{CC}$/V	&	$V_C$/V	&	$V_E$/V	&	$V_B$/V	\\ \midrule
12.01	&	7.27	&	1.913	&	2.577	\\ \bottomrule
    \end{tabular}
    \label{tab:price}
  \end{center}
\end{table}

 
 
 \begin{table}[htb]
  \begin{center}
    \caption{動作点の計算値}
    \begin{tabular}{cccc} \toprule
$V_{BE}$/V	&	$I_B$/$\mu$A	&	$V_{CE}$/V	&	$I_C$/mA	\\ \midrule
0.664	&	12.5	&	5.357	&	5.773	\\ \bottomrule
    \end{tabular}
    \label{tab:price}
  \end{center}
  \end{table}
  
  
  実験値から計算して求めた動作点を図\ \ に$Q_R$として示した。同図には準備課題で求めた動作点$Q_D$と、実測した動作点を$Q_M$に載せてある。
  
  動作点は図の中央にくることが望ましいという条件の元で作った図に対して、その理想点(5,6)から全体スケールに対して5\verb|%|以内の差を持つのみに収まったのは
  設計した増幅器の動作点が設計どおりになっているといえる数値であろう。
  
  
  
  ここでコンデンサ$C_E$の等価直列抵抗$R_{ESR}$が増幅器に与える影響について考える。図\ \ を見てみると、設計では10Vとしていた$V_{CC}'$が実測値から求めた交流負荷線では
  10.2Vとなっている。高周波では出力側はコンデンサと$R_C$との直列インピーダンスと見ることができるため、この差分が$I_{CQ}R_{ESR}$と等しいことを考えると、
  \begin{equation}
   R_{ESR} = 33 \Omega \nonumber
  \end{equation}
  となる。これは直列につながれたと考えられる$R_C=820\Omega$に比べて、約4\verb|%|となる。したがって出力インピーダンスからは無視できるほどコンデンサの等価直列抵抗は小さい、と言える。

 
   
   \clearpage
  
  \subsection{伝達特性}
 
   \subsubsection{原理}
   
   増幅器の回路に入力信号$v_i(V_2)$を入れたとき、出力信号$v_o$の値は
   \begin{equation}
    v_o = A_V v_i \\
   \end{equation}
   となるはずである。
   
   伝達特性の概形を示した図\ \ を参照されたい。
   入力と出力の関係は上で書いた通りになる。ただしある一定値を過ぎると増幅率が下がり飽和し始める。この領域はトランジスターの特性によるもので
   活性領域と呼ばれる([1])。
   
   入力電圧と位相の関係は次章(周波数特性)で述べるが、コンダクタを入れたことによりこの回路はハイパスフィルタとしての機能を持っている。今回の周波数(1kHz)では
   高周波数(パス領域)となるため、この領域ではコンダクタの機能を無視できるようになる。したがって抵抗のみになり入力に対して位相角は逆、つまり$\pi$のズレとなっている。
   
   また図\ \ に示すように出力波形は定められる。したがって、入出力関係が飽和し始めると、基準に対して下方向の波形だけ歪みが生じて見える。
   したがってこの歪みは設計によれば飽和し始めの電圧$v_o=$5Vで見られ始めるはずである([2])。
   
   \newpage
  
   \subsubsection{測定手順}
  
  \begin{enumerate}
   \item FGをオフセット電圧を0にして、周波数1kHzで出力。
   \item 入力電圧$V_1$を0Vから増加させていき、出力電圧が飽和し始まるまでを測定。飽和し始めの入力電圧を$v_i,max(pp)$、出力電圧を$v_o,max(pp)$。
   \item 直線上、飽和し始め、飽和状態、の3つの状態についてそれぞれ波形イメージを取得。
  \end{enumerate}

  
  
   \subsubsection{結果と考察}
   
 
 \begin{table}[htb]
  \begin{center}
    \caption{入力電圧$v_i$と出力電圧$v_o$}
    \begin{tabular}{cc} \toprule
$v_i$/mV	&	$v_o$/V	\\ \midrule
9.12	&	1.33	\\
18.2	&	2.66	\\
28.2	&	3.96	\\
39.2	&	5.20	\\
50.4	&	6.48	\\
64.8	&	7.60	\\
83.2	&	8.64	\\
106	&	9.26	\\ \bottomrule
    \end{tabular}
    \label{tab:price}
  \end{center}
\end{table}

 入出力特性に関しては図\ \ に示した。この図によれば$v_o=$5V付近で飽和し始め、その飽和点は入出力波形(図\ \ )からも歪みという形で確認することができる。
  未飽和領域(図\ \ )では歪みはなし、飽和し始め(図\ \ )では歪みが1.4倍程度に見られ、飽和状態(図\ \ )では1.8倍の歪みがあることが確認できた。
  
  したがって今回は
  \begin{eqnarray}
   飽和電位v_o &=& 5.2 \verb|V| \nonumber \\
   5Vからの相対誤差 &=& \frac{5.2 - 5.0}{5.0} \time 100 = 4.0 \verb|%| \nonumber \\
  \end{eqnarray}
  となる。
  
  また、図\ \ より入出力電圧増幅率$A_V$は
  \begin{eqnarray}
   A_V &\cong& 140 \nonumber \\
   理想値150からの相対誤差 &=& \frac{150 - 140}{150} = 6.7 \verb|%| \nonumber
  \end{eqnarray}
  となった。


  \clearpage


  また、入出力波形をみてみると、図\ ,\ ,\ ともに逆位相となっていることがわかる。
 交流負荷線、直流負荷線、動作点を示した図\ \ をみてみると、3つの動作点はすべて互いに多少の誤差はあるものの、定性的に、確かに交流負荷線に沿っている。
 これは増幅器の測定回路における$R_E$へ流れる電流は、高周波交流により並列接続されたコンデンサーにより遮断され、回路図から無視することができる、ということを意味している。
 すなわち出力側のインピーダンスは$R_C,C_E$を直列に繋いだ場合と等価となる。したがって高周波により、位相が反転していることが説明できる。



 
   
   \clearpage
  
  \subsection{周波数特性}
 
   \subsubsection{原理}
   $A_V-f,\phi-f$の関係式の導出を付録3に載せておくため、詳しくはそこを参照されたい。結論は次に載せておく。
 \begin{eqnarray}
  A_V &=& \sqrt{\frac{h_{fe}^2 R_C^2(1 + \omega^2 C_E^2 R_E^2)}{(h_{ie}+R_E h_{fe})^2 + (h_{ie} \omega C_E R_E)^2}} \nonumber \\
  \phi &=& \tan^{-1}(\frac{\omega C_E R_E (2h_{ie} + R_E h_{fe})}{h_{he} + R_E h_{fe} - h_{ie} \omega^2 C_E^2 R_E^2}) \nonumber
  \end{eqnarray}
  
  また、増幅器の消費電力$P_{CC}$、出力信号電力$P_O$、コレクタ損失$P_C$は次のように与えられる。
  \begin{eqnarray}
   P_{CC} &=& I_{CQ}V_{CC} \\
   P_C &=& I_{CQ}V_{CEQ} - \frac{1}{2} I_{Cm}V_{CEm} \\
   P_O &=& \frac{1}{2} I_{Cm}^2 R_C
  \end{eqnarray}

   
  
   \subsubsection{測定手順}
  
  \begin{enumerate}
   \item FGのオフセットを0にして、出力振幅を$v_{FG},max(pp)/2$程度となるように出力。
   \item 周波数10,20,50,100,200,500Hz、また1,2,5,10,20,50,100kHzについて入力信号の電圧$v_i(pp)$、出力信号の電圧$v_o(pp)$と位相差を測定。
  \end{enumerate}

  
  \clearpage
  
   \subsubsection{結果と考察}
   
 
 \begin{table}[htb]
  \begin{center}
    \caption{周波数特性}
    \begin{tabular}{cccccc} \toprule
f/kHz	&	$v_i$/mV	&	$v_o$/V	&	Av	&	$\Delta$t/ms	&	位相差/°	\\ \midrule
0.01	&	122	&	2.08	&	17.049	&	28	&	-100.8	\\
0.02	&	101	&	3.32	&	32.871	&	15	&	-108.0	\\
0.05	&	56.8	&	4.04	&	71.127	&	6.5	&	-117.0	\\
0.1	&	39.2	&	4.20	&	107.14	&	3.68	&	-132.5	\\
0.2	&	33.2	&	4.28	&	128.92	&	2.1	&	-151.2	\\
0.5	&	31.2	&	4.28	&	137.18	&	0.93	&	-167.4	\\
1	&	30.8	&	4.28	&	138.96	&	0.48	&	-172.8	\\
2	&	30.4	&	4.32	&	142.11	&	0.242	&	-174.2	\\
5	&	30.4	&	4.28	&	140.79	&	0.1	&	-180.0	\\
10	&	30.2	&	4.32	&	143.05	&	0.05	&	-180.0	\\
20	&	30.2	&	4.28	&	141.72	&	0.0248	&	-178.6	\\
50	&	30.0	&	4.24	&	141.33	&	0.01	&	-180.0	\\
100	&	28.8	&	4.08	&	141.67	&	0.00504	&	-181.4	\\ \bottomrule
    \end{tabular}
    \label{tab:price}
  \end{center}
\end{table}


 \clearpage

  表\ \ に実験データを、図\ \ に$A_V$、$\phi$の周波数特性を示した。

 図\ \ には理論式\ \ による理論線も載せているが、増幅率の大きさ$A_V$は周波数が大きくなるほどに、位相角$\phi$は周波数が低いほど、
 実測と理論の差が大きくなる。たとえば$A_V,\phi$の相対誤差の例を示すと、図\ \ より、

 \begin{table}[htb]
  \begin{center}
    \caption{$A_V,\phi$の相対誤差の周波数特性}
    \begin{tabular}{|c|c|c|} \hline
周波数 & 小 & 大 \\ \hline
$A_V$ & 0\verb|%| & 4\verb|%| \\ \hline
$\phi$ & 11\verb|%| & 0\verb|%| \\ \hline
    \end{tabular}
    \label{tab:price}
  \end{center}
 \end{table}
  
  となる。
  
  これを見ても分かるように、増幅率の大きさと位相角ではその相対誤差の周波数特性が異なっている。
  
  位相差については、周波数が大きくなるとトランジスタCE間(出力側)インピーダンスの実数部分が大きな割合を占めるため(伝達特性原理、参照)、
  周波数が小さいときに大きく存在した誤差も、周波数が大きくなるに連れて、理論値と実測値も漸近値である180°に近づき、互いの誤差が小さくなる。
  
  また増幅率の大きさに関しては、周波数が小さくなるにつれて増幅率は0に近づく。その全体量の小ささ故に、低周波数域では誤差がほとんど現れてこない。しかし、
  高周波域になると互いに量が大きくなるため互いの差も全体スケールに対して現れ始める。加えて、理論値が設計値の150に近づいているのに対し、実測値がそれを7\verb|%|ほど下回ってる
  ことが理由に考えられる。実験精度の問題だが、理論値との誤差が4\verb|%|となったのは、小信号等価回路という形の近似回路で計算したものとして考えるなら、高い精度だと考えられる。
  
  高周波でも低周波でも、増幅度の大きさか位相角どちらかは相対誤差は大きくなっているが、一方必ず片方は高い近似が得られている点で、この小信号等価回路の増幅度計算は有効であることが
  確かめられた。
   
   \clearpage
  
 \section{レポート課題}
 \subsection{設計増幅器の仕様との差異}
  ここで設計した回路の実際の仕様を確認すると、
 \begin{itemize}
  \item 1kHz以上の周波数において電圧利得一定(平均値からの誤差1%未満)。
  \item Av $141 \pm 1$
  \item $v_o$:飽和電圧5.2V,誤差4\verb|%|(伝達特性参照)
 \end{itemize}
  したがって使用の範囲は満たしていると十分にいえる。
  
 \subsection{ベース回路のダイオード特性}
 直流特性の原理、考察でも述べた通り、トランジスタのBE接合、つまりpn接合はダイオードのpn接合と同様の特性を持つ、その証拠に今回の直流特性の実験結果にあるように$I_B-V_{BE}$にはダイオードの
 特性が見られ、$ln(I_B)-V_{BE}$図は直線近似し、高い結晶性を示す理想係数が傾きから求めることができた。また一般的に、Siダイオードの閾値は約0.7Vである。
 
 したがって、トランジスタのベース電圧の閾値が0.7Vと予測できるのはトランジスタのBE間が示す、ダイオード特性によるものである。
 
 \subsection{動作点の確認}
 今回の実験値から計算した動作点を図\ \ に示した。動作点は(5.1V,6.2mA)となっている。
 図\ \ に示した通り、3つの動作点は(計算、実測、設計)すべて交流負荷線に沿って分布している。これはコンデンサ$C_E$により$R_E$が無視されていることを示し、また
 電圧的な動作点の誤差が5\verb|%|以内に収まってることは、この実験の理想の設計動作点が(5V,6mA)であることから、設計どおりであるといえる。
 
 \subsection{伝達特性での入力信号と出力信号の位相差}
 伝達特性の原理にもあるように、1kHzほどの高周波だと増幅器特性回路図から$R_E$が無視できる、かつ
 \begin{eqnarray}
  Z &=& \frac{\omega RC - j}{\omega C} \nonumber \\
  \omega RC &>>& -1 より、 \nonumber \\
  \phi &\cong& \pm \pi \nonumber
 \end{eqnarray}
 より、図\ \ ,\ \ ,\ \ で位相が逆になっていることが分かる。
 
 \subsection{周波数特性}
 表\ \ に示すように、また伝達特性の考察でも考えたように位相差は高周波になると理論値・実測値ともに180°に近づくため、相対誤差は小さくなる。
 
 増幅率の大きさは、低周波では理論値、実測値ともに値が小さいため、誤差は現れてこない。また周波数が高くなると、実験値精度の悪さからくる実測値の設計値からの差異によって、理論値からの
 相対誤差は大きくなっている。
 
 \subsection{コンデンサ$C_E$の等価直列抵抗が増幅器の特性に与える影響}
 回路製作・動作点の確認の考察より、コンデンサの等価直列抵抗値は直列に接続されたとみなせる$R_C=820\Omega$にくらべて、4\verb|%|程度である。したがって、
 出力インピーダンスからは無視できるほどコンデンサの等価直列抵抗は小さい、と言える。
 
 \subsection{$h_{oe}$が増幅器の特性に与える影響}
 交流特性の考察より、$h_{oe}$パラメータの増幅器に与える影響は$h_{fe}$の与える影響の2\verb|%|に過ぎない。したがって、$h_{oe}$パラメータは無視できるほどに小さいと言うことができる。
 
 \subsection{消費電力$P_{CC}$、出力信号電力$P_O$、コレクタ損失$P_C$}
 
  \begin{enumerate}
   \item 無信号のとき \\
   $P_O = 0W, P_{CC} = 74.4mW, P_C = 31.62 mW$ \\
   \item 正弦波信号を入力し、歪みのない最大出力振幅が得られているとき \\
   $P_O = 3.48mW, P_{CC} = 74.mW, P_C = 28.1mW$ \\
  \end{enumerate}

  
  
 \section{まとめ}
 
  \clearpage
  
  
 \section{参考文献}
 [1]古川静二郎他著、「電子デバイス工学 第2版」、森北出版株式会社
 
 [2]\verb|http://www.kairo-nyumon.com/practice_summary.html|
  
  
\end{document}