\documentclass[11pt,a4j]{jsarticle}
\title{磁気ヒステリシス特性}
\author{1413176 三村幸祐}
\date{2016/6/16 \and 2016/6/23}
\usepackage{booktabs}

\begin{document}
  
  
 \section{目的}
  変圧器の鉄心の履歴現象を調べる。無負荷試験と磁気ヒステリシス特性試験からそれぞれ損失を求める。2通りの方法で計算したこの同じ物理量を比較検討する。
  
 \section{無負荷試験}
 
  \subsection{原理}
   \subsubsection{変圧器のアドミタンス}
   \begin{equation}
    Y_0 = g_0 -jb_0 \\
   \end{equation}
   
    ここで等価回路のアドミタンス$Y_0$を考えると、
    \begin{equation}
    Y_0 = 1/R - j 1/{\omega L} \\
   \end{equation}
  
   したがって、$g_0 = 1/R$、また$b_0 = 1/{\omega L}$である。

   変圧器で消費される電力をP、変圧器にかかる電圧の実効値をV、流れる電流の実効値をIとすると、
   \begin{eqnarray}
    P_R &=& \frac{V^2}{R} cos\theta \nonumber \\
        &=& \frac{V^2}{R} \nonumber \\
    P_L &=& \frac{V^2}{j\omega L} cos\theta \nonumber \\
        &=& 0 \nonumber \\
    P   &=& P_R + P_L \nonumber \\
        &=& \frac{V^2}{R}
   \end{eqnarray}
   ただし、$\theta$はR,Lの各インピーダンスが虚数空間上で実数軸とのなす角とする。

   したがって、
   \begin{eqnarray}
    g_0 &=& 1/R \nonumber \\
        &=& P/V^2 \\
    b_0 &=& \sqrt{Y_0^2 - g_0^2} \nonumber \\ 
        &=& \frac{1}{V} \sqrt{I^2 - \frac{P^2}{V^2}} 
   \end{eqnarray}
   となる。
   
   力率も考えてみると、
   \begin{equation}
    力率P\cos{\theta} = P \frac{P}{V^2} \frac{V}{I} = \frac{P^2}{VI} 
   \end{equation}
   で与えられる。
   
   ここでの$\theta$はR,Lの等価回路におけるアドミタンスが虚数空間上で実数軸となす角である。
   
   \subsubsection{変圧器に流れる電流}
   変圧器の等価回路から計算する電流は、その等価回路に流れる電流$i_0$を考えればよい。R,Lが前項と同様、並列に接続されるため、電流$i_0$の内、
   R成分に流れる電流を$i_R$、L成分に流れる電流を$i_L$とおくと、$i_0 = i_R + i_L$となることがわかる(キルヒホッフの電流則)。また電源の電圧は、
   $v = \sqrt{2} V sin\omega t$、電圧$v$と電流$i_0$の位相差を$\theta$とすると、
   \begin{eqnarray}
    i_R &=& v/R = \sqrt{2} \frac{V}{R} sin\omega t \\
    i_L &=& \frac{1}{L}\int vdt = -\sqrt{2}\frac{V}{\omega L} cos\omega t \\
    i_0 &=& i_R + i_L \nonumber \\
        &=& \sqrt{2} V \frac{1}{\omega RL} (\omega Lsin\omega t - R cos\omega t) \nonumber \\
        && cos\theta = \frac{\omega L}{\sqrt{R^2 + (\omega L)^2}} \nonumber \\
        && sin\theta = \frac{R}{\sqrt{R^2 + (\omega L)^2}} より、 \nonumber \\
        &=& \frac{\sqrt{2}V}{\omega RL} \sqrt{(\omega L)^2 + R^2} sin(\omega t - \theta) \nonumber \\
        &=& \sqrt{2} V \sqrt{g_0^2 + b_0^2} sin(\omega t - \theta) \nonumber \\
        &=& \sqrt{2} I sin(\omega t - \theta)
   \end{eqnarray}
   となる。


  \subsection{結果と考察}
  無負荷試験のデータを表1に、それらをまとめた各2物理量の特性グラフを図1,2,3,4に示す。
  
  また電流波形を図5に、電圧の積分波形を図6に、そして起磁力$F=Ni_0$と$\phi = \frac{1}{N} \int v dt$の関係を図7に示した。

  
 \begin{table}[htb]
  \begin{center}
    \caption{無負荷試験のデータ}
    \begin{tabular}{cccccccc} \toprule
電圧[V]rms	&	電流[A]rms	&	電力[W]	&	電力(読値)	&	go[S]	&	bo[S]	&	go[mS]	&	bo[mS]	\\ \midrule
50	&	0.100 	&	3.708	&	18.0	&	0.0014832	&	0.001342 	&	1.483 	&	1.342 	\\
55	&	0.110 	&	4.326	&	21.0	&	0.0014301	&	0.001398 	&	1.430 	&	1.398 	\\
60	&	0.125 	&	5.150	&	25.0	&	0.0014306	&	0.001515 	&	1.431 	&	1.515 	\\
65	&	0.135 	&	5.871	&	28.5	&	0.0013896	&	0.001544 	&	1.390 	&	1.544 	\\
70	&	0.155 	&	6.695	&	32.5	&	0.0013663	&	0.001742 	&	1.366 	&	1.742 	\\
75	&	0.180 	&	7.725	&	37.5	&	0.0013733	&	0.001968 	&	1.373 	&	1.968 	\\
80	&	0.215 	&	8.755	&	42.5	&	0.0013680	&	0.002313 	&	1.368 	&	2.313 	\\
85	&	0.251 	&	9.991	&	48.5	&	0.0013828	&	0.002609 	&	1.383 	&	2.609 	\\
90	&	0.295 	&	11.124	&	54.0	&	0.0013733	&	0.002976 	&	1.373 	&	2.976 	\\
95	&	0.350 	&	12.566	&	61.0	&	0.0013924	&	0.003411 	&	1.392 	&	3.411 	\\
100	&	0.420 	&	14.008	&	68.0	&	0.0014008	&	0.003960 	&	1.401 	&	3.960 	\\ \bottomrule
    \end{tabular}
    \label{tab:price}
  \end{center}
\end{table}
  
 \section{磁気ヒステリシス}
 
  \subsection{原理}
  \subsubsection{損失Wと有効電力Pの関係}
  起磁力対磁束曲線で囲まれた面積(損失W)と電力計で測定される値(有効電力)の関係を考える。
  
  まず、電力計で計測される平均有効電力の値は
  \begin{equation}
   P = \frac{1}{T} \int_0^T i_1 v dt \\
  \end{equation}
  
  起磁力対磁束曲線で囲まれた面積(損失)は
  \begin{equation}
   W = \oint F d\phi \\
  \end{equation}
  で表される。
  
  したがってここから$P,W$の関係を考えると、
  \begin{eqnarray}
   W &=& N i_1 d\phi \nonumber \\
     &=& N \int i_1 \frac{1}{N} v dt \nonumber \\
     &=& T \cdot P
  \end{eqnarray}
  となる。

  \subsubsection{磁束$\phi$の関係式}
   
   \begin{eqnarray}
    Z &=& \sqrt{R^2 + \frac{1}{\omega^2 C^2}} \nonumber \\
      &=& \sqrt{\frac{\omega^2 R^2 C^2 + 1}{\omega^2 C^2}} \nonumber \\
      &=& \frac{1}{C} \sqrt{(RC)^2 + (\frac{T}{2\pi})^2} \nonumber \\
      &\cong& R
   \end{eqnarray}
   
   ただし、$CR\gg T = 1/f$により近似している。

   したがって、
   \begin{eqnarray}
    \phi &=& \frac{1}{N_2} \int Ri_2 dt \nonumber \\
         && \int i_2 dt = ce_2 より、 \nonumber \\
    \phi &=& \frac{RC}{N_2} e_2
   \end{eqnarray}
    となる。
  

  \subsection{結果と考察}
  
    
  各電圧ごとの鉄心の磁気特性を表2,3,4に示す。また$\phi-F$グラフを図8(V=100V),図9(V=80V),図10(V=60V)に示す。
  
   \begin{table}[htb]
  \begin{center}
    \caption{100Vの時の鉄心の磁気特性}
    \begin{tabular}{cccccc} \toprule
時間[ms]	&	e1 [mV]	&	i1[A]	&	F[A]	&	e2 [mV]	&	φ[wb]	\\ \midrule
0.00	&	-146.00	&	-0.87	&	-119.77		&	-182.50	&	-3.318E-03	\\
0.50	&	-132.00	&	-0.79	&	-108.29		&	-181.20	&	-3.294E-03	\\
1.00	&	-103.30	&	-0.62	&	-84.74		&	-175.70	&	-3.194E-03	\\
1.50	&	-70.50	&	-0.42	&	-57.84		&	-166.40	&	-3.025E-03	\\
2.00	&	-42.30	&	-0.25	&	-34.70		&	-154.20	&	-2.803E-03	\\
2.50	&	-16.40	&	-0.10	&	-13.45		&	-137.30	&	-2.496E-03	\\
3.00	&	0.25	&	0.00	&	0.21		&	-115.50	&	-2.100E-03	\\
3.50	&	12.50	&	0.07	&	10.25		&	-92.50	&	-1.682E-03	\\
4.00	&	17.10	&	0.10	&	14.03		&	-69.30	&	-1.260E-03	\\
4.50	&	19.70	&	0.12	&	16.16		&	-41.90	&	-7.617E-04	\\
5.00	&	22.10	&	0.13	&	18.13		&	-13.80	&	-2.509E-04	\\
5.50	&	24.90	&	0.15	&	20.43		&	14.40	&	2.618E-04	\\
6.00	&	28.20	&	0.17	&	23.13		&	42.60	&	7.744E-04	\\
6.50	&	32.70	&	0.20	&	26.83		&	71.50	&	1.300E-03	\\
7.00	&	38.40	&	0.23	&	31.50		&	96.40	&	1.752E-03	\\
7.50	&	47.40	&	0.28	&	38.89		&	118.60	&	2.156E-03	\\
8.00	&	64.20	&	0.38	&	52.67		&	138.40	&	2.516E-03	\\
8.50	&	88.60	&	0.53	&	72.68		&	155.80	&	2.832E-03	\\
9.00	&	114.50	&	0.69	&	93.93		&	168.70	&	3.067E-03	\\
9.50	&	134.50	&	0.81	&	110.34		&	177.70	&	3.230E-03	\\
10.00	&	142.50	&	0.85	&	116.90		&	181.20	&	3.294E-03	\\
10.50	&	133.50	&	0.80	&	109.52		&	179.40	&	3.261E-03	\\
11.00	&	100.30	&	0.60	&	82.28		&	174.20	&	3.167E-03	\\
11.50	&	70.30	&	0.42	&	57.67		&	164.50	&	2.990E-03	\\
12.00	&	38.60	&	0.23	&	31.67		&	151.40	&	2.752E-03	\\
12.50	&	12.40	&	0.07	&	10.17		&	135.30	&	2.460E-03	\\
13.00	&	-3.70	&	-0.02	&	-3.04		&	117.30	&	2.132E-03	\\
13.50	&	-13.30	&	-0.08	&	-10.91		&	94.60	&	1.720E-03	\\
14.00	&	-17.90	&	-0.11	&	-14.68		&	69.10	&	1.256E-03	\\
14.50	&	-20.60	&	-0.12	&	-16.90		&	42.10	&	7.653E-04	\\
15.00	&	-23.00	&	-0.14	&	-18.87		&	15.50	&	2.818E-04	\\
15.50	&	-25.70	&	-0.15	&	-21.08		&	-13.30	&	-2.418E-04	\\
16.00	&	-29.10	&	-0.17	&	-23.87		&	-42.30	&	-7.689E-04	\\
16.50	&	-33.60	&	-0.20	&	-27.56		&	-70.40	&	-1.280E-03	\\
17.00	&	-39.10	&	-0.23	&	-32.08		&	-96.20	&	-1.749E-03	\\
17.50	&	-48.60	&	-0.29	&	-39.87		&	-118.50	&	-2.154E-03	\\
18.00	&	-64.80	&	-0.39	&	-53.16		&	-137.70	&	-2.503E-03	\\
18.50	&	-87.20	&	-0.52	&	-71.54		&	-154.20	&	-2.803E-03	\\
19.00	&	-113.50	&	-0.68	&	-93.11		&	-167.70	&	-3.049E-03	\\
19.50	&	-134.60	&	-0.81	&	-110.42		&	-176.50	&	-3.208E-03	\\
20.00	&	-145.20	&	-0.87	&	-119.12		&	-180.20	&	-3.276E-03	\\ \bottomrule
    \end{tabular}
    \label{tab:price}
  \end{center}
\end{table}

 \begin{table}[htb]
  \begin{center}
    \caption{80Vの時の鉄心の磁気特性}
    \begin{tabular}{cccccc} \toprule
時間[ms]	&	e1 [mV]	&	i1[A]	&	F[A]	&	e2 [mV]	&	φ[wb]	\\ \midrule
0.00	&	-66.80	&	-0.40	&	-54.80		&	-143.60	&	-2.610E-03	\\
0.50	&	-60.50	&	-0.36	&	-49.63		&	-143.30	&	-2.605E-03	\\
1.00	&	-45.50	&	-0.27	&	-37.33		&	-139.60	&	-2.538E-03	\\
1.50	&	-29.70	&	-0.18	&	-24.36		&	-132.60	&	-2.410E-03	\\
2.00	&	-13.20	&	-0.08	&	-10.83		&	-122.70	&	-2.230E-03	\\
2.50	&	-1.20	&	-0.01	&	-0.98		&	-109.20	&	-1.985E-03	\\
3.00	&	7.60	&	0.05	&	6.23		&	-92.60	&	-1.683E-03	\\
3.50	&	12.70	&	0.08	&	10.42		&	-74.90	&	-1.362E-03	\\
4.00	&	15.80	&	0.09	&	12.96		&	-55.40	&	-1.007E-03	\\
4.50	&	17.70	&	0.11	&	14.52		&	-33.70	&	-6.126E-04	\\
5.00	&	19.50	&	0.12	&	16.00		&	-11.30	&	-2.054E-04	\\
5.50	&	21.40	&	0.13	&	17.56		&	11.30	&	2.054E-04	\\
6.00	&	23.60	&	0.14	&	19.36		&	34.10	&	6.199E-04	\\
6.50	&	26.20	&	0.16	&	21.49		&	56.60	&	1.029E-03	\\
7.00	&	28.90	&	0.17	&	23.71		&	77.00	&	1.400E-03	\\
7.50	&	32.60	&	0.20	&	26.74		&	95.30	&	1.732E-03	\\
8.00	&	38.00	&	0.23	&	31.17		&	110.90	&	2.016E-03	\\
8.50	&	46.10	&	0.28	&	37.82		&	124.50	&	2.263E-03	\\
9.00	&	55.60	&	0.33	&	45.61		&	135.70	&	2.467E-03	\\
9.50	&	63.20	&	0.38	&	51.85		&	142.50	&	2.590E-03	\\
10.00	&	65.60	&	0.39	&	53.82		&	145.40	&	2.643E-03	\\
10.50	&	60.30	&	0.36	&	49.47		&	144.30	&	2.623E-03	\\
11.00	&	45.30	&	0.27	&	37.16		&	141.00	&	2.563E-03	\\
11.50	&	29.20	&	0.17	&	23.95		&	134.00	&	2.436E-03	\\
12.00	&	12.40	&	0.07	&	10.17		&	123.10	&	2.238E-03	\\
12.50	&	0.40	&	0.00	&	0.33		&	109.50	&	1.991E-03	\\
13.00	&	-8.20	&	-0.05	&	-6.73		&	93.20	&	1.694E-03	\\
13.50	&	-13.50	&	-0.08	&	-11.07		&	74.20	&	1.349E-03	\\
14.00	&	-16.50	&	-0.10	&	-13.54		&	55.10	&	1.002E-03	\\
14.50	&	-18.40	&	-0.11	&	-15.09		&	34.20	&	6.217E-04	\\
15.00	&	-20.20	&	-0.12	&	-16.57		&	12.30	&	2.236E-04	\\
15.50	&	-22.10	&	-0.13	&	-18.13		&	-10.10	&	-1.836E-04	\\
16.00	&	-24.20	&	-0.14	&	-19.85		&	-33.30	&	-6.053E-04	\\
16.50	&	-26.70	&	-0.16	&	-21.90		&	-56.60	&	-1.029E-03	\\
17.00	&	-29.50	&	-0.18	&	-24.20		&	-77.30	&	-1.405E-03	\\
17.50	&	-33.00	&	-0.20	&	-27.07		&	-95.40	&	-1.734E-03	\\
18.00	&	-38.30	&	-0.23	&	-31.42		&	-112.10	&	-2.038E-03	\\
18.50	&	-45.30	&	-0.27	&	-37.16		&	-124.50	&	-2.263E-03	\\
19.00	&	-55.10	&	-0.33	&	-45.20		&	-135.80	&	-2.469E-03	\\
19.50	&	-64.40	&	-0.39	&	-52.83		&	-143.70	&	-2.612E-03	\\
20.00	&	-65.90	&	-0.39	&	-54.06		&	-146.50	&	-2.663E-03	\\ \bottomrule
    \end{tabular}
    \label{tab:price}
  \end{center}
\end{table}

 \begin{table}[htb]
  \begin{center}
    \caption{60Vの時の鉄心の磁気特性}
    \begin{tabular}{cccccc} \toprule
時間[ms]	&	e1 [mV]	&	i1[A]	&	F[A]	&	e2 [mV]	&	φ[wb]	\\ \midrule
0.0 	&	-30.4 	&	-0.18	&	-24.94		&	-109.4 	&	-1.99E-03	\\
0.5 	&	-26.6 	&	-0.16	&	-21.82		&	-109.0 	&	-1.98E-03	\\
1.0 	&	-21.0 	&	-0.13	&	-17.23		&	-105.7 	&	-1.92E-03	\\
1.5 	&	-12.8 	&	-0.08	&	-10.50		&	-100.4 	&	-1.83E-03	\\
2.0 	&	-3.7 	&	-0.02	&	-3.04		&	-89.7 	&	-1.63E-03	\\
2.5 	&	1.8 	&	0.01	&	1.48		&	-82.4 	&	-1.50E-03	\\
3.0 	&	7.4 	&	0.04	&	6.07		&	-69.5 	&	-1.26E-03	\\
3.5 	&	11.0 	&	0.07	&	9.02		&	-53.8 	&	-9.78E-04	\\
4.0 	&	13.4 	&	0.08	&	10.99		&	-42.3 	&	-7.69E-04	\\
4.5 	&	15.0 	&	0.09	&	12.31		&	-25.5 	&	-4.64E-04	\\
5.0 	&	16.3 	&	0.10	&	13.37		&	-9.2 	&	-1.67E-04	\\
5.5 	&	17.7 	&	0.11	&	14.52		&	7.5 	&	1.36E-04	\\
6.0 	&	19.0 	&	0.11	&	15.59		&	24.6 	&	4.47E-04	\\
6.5 	&	20.4 	&	0.12	&	16.74		&	42.1 	&	7.65E-04	\\
7.0 	&	21.5 	&	0.13	&	17.64		&	56.8 	&	1.03E-03	\\
7.5 	&	23.1 	&	0.14	&	18.95		&	70.7 	&	1.29E-03	\\
8.0 	&	25.0 	&	0.15	&	20.51		&	82.9 	&	1.51E-03	\\
8.5 	&	26.8 	&	0.16	&	21.99		&	92.8 	&	1.69E-03	\\
9.0 	&	28.8 	&	0.17	&	23.63		&	100.4 	&	1.83E-03	\\
9.5 	&	29.8 	&	0.18	&	24.45		&	105.5 	&	1.92E-03	\\
10.0 	&	29.6 	&	0.18	&	24.28		&	108.5 	&	1.97E-03	\\
10.5 	&	26.6 	&	0.16	&	21.82		&	108.1 	&	1.97E-03	\\
11.0 	&	20.0 	&	0.12	&	16.41		&	104.7 	&	1.90E-03	\\
11.5 	&	11.5 	&	0.07	&	9.43		&	98.6 	&	1.79E-03	\\
12.0 	&	3.7 	&	0.02	&	3.04		&	91.2 	&	1.66E-03	\\
12.5 	&	-3.0 	&	-0.02	&	-2.46		&	80.7 	&	1.47E-03	\\
13.0 	&	-8.1 	&	-0.05	&	-6.64		&	68.8 	&	1.25E-03	\\
13.5 	&	-11.7 	&	-0.07	&	-9.60		&	55.3 	&	1.01E-03	\\
14.0 	&	-14.1 	&	-0.08	&	-11.57		&	40.3 	&	7.33E-04	\\
14.5 	&	-15.8 	&	-0.09	&	-12.96		&	24.8 	&	4.51E-04	\\
15.0 	&	-17.1 	&	-0.10	&	-14.03		&	7.6 	&	1.38E-04	\\
15.5 	&	-18.4 	&	-0.11	&	-15.09		&	-9.3 	&	-1.69E-04	\\
16.0 	&	-19.7 	&	-0.12	&	-16.16		&	-26.4 	&	-4.80E-04	\\
16.5 	&	-21.1 	&	-0.13	&	-17.31		&	-43.8 	&	-7.96E-04	\\
17.0 	&	-22.3 	&	-0.13	&	-18.29		&	-58.6 	&	-1.07E-03	\\
17.5 	&	-23.7 	&	-0.14	&	-19.44		&	-72.0 	&	-1.31E-03	\\
18.0 	&	-25.4 	&	-0.15	&	-20.84		&	-83.4 	&	-1.52E-03	\\
18.5 	&	-27.5 	&	-0.16	&	-22.56		&	-93.2 	&	-1.69E-03	\\
19.0 	&	-29.3 	&	-0.18	&	-24.04		&	-100.3 	&	-1.82E-03	\\
19.5 	&	-30.3 	&	-0.18	&	-24.86		&	-105.9 	&	-1.93E-03	\\
20.0 	&	-29.4 	&	-0.18	&	-24.12		&	-108.1 	&	-1.97E-03	\\ \bottomrule
    \end{tabular}
    \label{tab:price}
  \end{center}
\end{table}
  
 \section{参考文献}
  
  
  
\end{document}