\documentclass[11pt,a4j]{jsarticle}
\title{デジタルの準備課題1}
\author{1413176 三村幸祐}
\date{}
\usepackage{booktabs}

\begin{document}
 
 \section{準備課題1-1}
 
 
 \begin{table}[htb]
  \begin{center}
    \caption{スイッチ操作とLEDの点滅の予測}
    \begin{tabular}{ccccccccccccccc} \toprule
SW0 & SW1 & P31 & P33 & A & B & X & P35 & LEDR2 & Y & P37 & LEDR3 & Z & P39 & LEDR4 \\ \midrule
on & on & H & H & 1 & 1 & 1 & H & 点 & 1 & H & 点 & 0 & L & 滅 \\
on & off & H & L & 1 & 0 & 0 & L & 滅 & 1 & H & 点 & 1 & H & 点 \\
off & on & L & H & 0 & 1 & 0 & L & 滅 & 1 & H & 点 & 1 & H & 点 \\
off & off & L & L & 0 & 1 & 0 & L & 滅 & 0 & L & 滅 & 0 & L & 滅 \\ \bottomrule
    \end{tabular}
    \label{tab:price}
  \end{center}
\end{table}

 \section{準備課題1-2}
 
 \begin{eqnarray}
  V_H &=& R i + \frac{q}{C} \nonumber \\
      &=& R \frac{d}{dt} q + \frac{1}{C} q \nonumber \\
  (\frac{d}{dt} + \frac{1}{RC}) q &=& \frac{V_H}{R} \nonumber \\
  q &=& CV_H + A e^{-\frac{1}{RC}t} (Aは任意の実数) \nonumber \\
  && q(0) = 0 より、 \nonumber \\
  q(0) &=& CV_H + A \nonumber \\
  A &=& -CV_H \nonumber \\
  q(t) &=& CV_H (1-e^{-\frac{1}{RC}t}) \nonumber \\ \nonumber \\
  V_2(t) &=& \frac{q(t)}{C} \nonumber \\
         &=& V_H (1-e^{-\frac{1}{RC}t}) \nonumber \\ \nonumber \\
  && V_2(t)=V_{TH} = 1.8Vの時、 \nonumber \\
  t &=& 0.788 \mu s \nonumber
 \end{eqnarray}

 \newpage
 
 \section{準備課題1-3}
 
 
 \begin{table}[htb]
  \begin{center}
    \caption{全加算回路の真理値表}
    \begin{tabular}{ccccc} \toprule
A & B & $C_{IN}$ & S & $C_{OUT}$ \\ \midrule
0 & 0 & 0 & 0 & 0 \\
0 & 0 & 1 & 1 & 0 \\
0 & 1 & 0 & 1 & 0 \\
0 & 1 & 1 & 0 & 1 \\
1 & 0 & 0 & 1 & 0 \\
1 & 0 & 1 & 0 & 1 \\
1 & 1 & 0 & 0 & 1 \\
1 & 1 & 1 & 1 & 1 \\ \bottomrule
    \end{tabular}
    \label{tab:price}
  \end{center}
\end{table}
 
 
 \section{準備課題2-1}
 
 
 \begin{table}[htb]
  \begin{center}
    \caption{RSフリップフロップの動作予測表}
    \begin{tabular}{ccccc} \toprule
R & S & $Q_(前)$ & $Q_(現)$ & $Q_(現)$ \\ midrule
0 & 1 & 0 & 0 & 1 \\
1 & 0 & 0 & 1 & 1 \\
0 & 0 & 1 & 1 & 0 \\
1 & 0 & 1 & 1 & 0 \\
1 & 1 & 1 & 1 & 1 \\
0 & 1 & 1 & 0 & 1 \\
0 & 0 & 0 & 0 & 1 \\
1 & 1 & 0 & 1 & 1 \\ \bottomrule
    \end{tabular}
    \label{tab:price}
  \end{center}
\end{table}
 
 
 
 
\end{document}
